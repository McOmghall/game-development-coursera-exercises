\documentclass[12pt]{article}
\usepackage[pdftex]{graphicx}
\usepackage{url}
\usepackage{setspace}
\usepackage{makeidx}
\usepackage[utf8]{inputenc}
\usepackage{fancyhdr}
\usepackage[a4paper,pdftex]{geometry}	% Use A4 paper margins
\usepackage[english,spanish]{babel}
\usepackage{xcolor} % Required for specifying custom colors
\usepackage{fix-cm} % Allows increasing the font size of specific fonts beyond LaTeX default specifications

\setlength{\oddsidemargin}{0mm} % Adjust margins to center the colored title box
\setlength{\evensidemargin}{0mm} % Margins on even pages - only necessary if adding more content to this template

\setlength{\voffset}{-1.2cm}
\setlength{\textheight}{650pt}
\setlength{\parindent}{0pt}
\renewcommand{\baselinestretch}{1.5}
\definecolor{grey}{rgb}{0.9,0.9,0.9} % Color of the box surrounding the title - these values can be changed to give the box a different color	

\pagestyle{fancy}
\fancyhf{}
\lhead{A Fireball for Your Friends}
\rhead{Concept Proposal Document}
\rfoot{\thepage}

\makeindex

\begin{document}

\thispagestyle{empty} % Remove page numbering on this page

%----------------------------------------------------------------------------------------
%	TITLE SECTION
%----------------------------------------------------------------------------------------

\colorbox{grey}{
	\parbox[t]{1.0\linewidth}{
		\fontsize{50pt}{30pt}\selectfont % The first argument for fontsize is the font size of the text and the second is the line spacing - you may need to play with these for your particular title
		\vspace*{0.7cm} % Space between the start of the title and the top of the grey box
		
		A Fireball \\ 
		for Your Friends \\ 
        \fontsize{30pt}{34pt}\selectfont
        Concept Proposal Document		
		\par
		
		\vspace*{0.4cm} % Space between the end of the title and the bottom of the grey box
	}
}

\vspace*{0.4cm} 
{\large Un \textbf{juego de duelos mágicos multijugador en 3ª persona}}

\begin{spacing}{0.6}
Target: \textit{chicos entre 16-22 años, amantes de los juegos competitivos, mid-core} 

Plataforma: \textit{Nintendo Switch (consola y portátil)}
\end{spacing}


\vfill % Space between the title box and author information

{\centering \hfill \copyright 2016 Pedro Montoto García} \\

%----------------------------------------------------------------------------------------

\clearpage % Whitespace to the end of the page

\printindex

\clearpage % Whitespace to the end of the page

\setlength{\voffset}{0cm}
\setlength{\parindent}{1cm}
\setcounter{page}{1}

\section{Concepto General}

Un juego multijugador (pantalla partida y online) en tercera persona donde dos equipos cada uno con de 1 a 3 magos duelean en un pequeño escenario usando hechizos variados y espectaculares. Cada jugador deberá componer su biblioteca de hechizos en cada encuentro, escogiéndolos según la utilidad que tengan para contrarrestar al enemigo y sinergizar con sus aliados. 

Cada encuentro tendrá sus objetivos, i.e. aniquilación, capturar la bandera, defender un objetivo, football..., y su duración será corta, i.e. entre 5 y 10 minutos, para asegurar que todo el mundo pueda disfrutar de al menos una partida rápida, pero que ésta sea intensa.

La experiencia de juego se centra en hacer que el jugador sienta su maestría al derrotar a enemigos usando los diferentes hechizos, evitándolos a su vez. El loop de juego principal estará en dominar los diferentes hechizos y sus sinergias via el empoderamiento del jugador.

\newpage

\section{Features and Scope}

MVP features:

\begin{enumerate}
 	\item[Local Multiplayer] Allowing people to get together in their homes and play your game is a very good method for creating a pleasant experience they'll remember.
	\item[3rd Person Action] 3D graphics and gameplay feedback elements must be adjusted to fit this kind of UX.
	\item[Hat Market] People want their wizard to be different from other people's, obviously.
\end{enumerate}

Extra features:

\begin{enumerate}
	\item[Online Play] Players connect to a server where they can fight other players from all over the globe. As a modern game this should be achieved through automated matchmaking but also giving the players the ability to fight people they know\footnote{As this is 1v1 some modified versions of ELO can work quite well}.
	\item[Training Mode] Play against configurable bots to test spells and your own skills.
	\item[Modding] Let people create their own modes, spells, maps, aesthetic content, etc, expose some game data to be easily changed to add them to the game
	\item[Leaderboards] Show everyone how good you are!
\end{enumerate}

AAA features:

\begin{enumerate}
	\item[Tournaments] Host tournaments for the best players. Stream them through in-game features: give people more reasons for being better.
	\item[Story Mode] A story mode that uses the already created gameplay elements can be easily integrated into the game, but it shouldn't be thought as just a wrapper for matches against bots (as usually seen in fighting and RTS games).
\end{enumerate}

\newpage
\section{State of the Art}
\subsection{The Competition}
This game competes directly with other shooters such as Call of Duty, Battlefield, S3 League (also 3rd person), MOBAs and other highly-competitive online games like DOTA2 and LOL, and also with medieval/magic themed 3rd person combat games like Dark Souls.

\subsection{Innovation/Product differentiation}
We believe this particular combination of highly competitive gaming, 3rd person shooting and magic theme hasn't been tried, therefore taking the maket is mostly a problem of getting a tight gameplay that creates tension with satisfactory resolution for all parts involved and using iconic art and sound.


\end{document}
